%\documentclass[aps,superscriptaddress,showpacs,nofootinbib,eqsecnum,prd,twocolumn]{revtex4-1}
\documentclass[aps,showpacs,nofootinbib,eqsecnum,prd,twocolumn]{revtex4-1}
\usepackage{epsfig}
\input{epsf}

\usepackage{graphicx}
%\usepackage[numbers,sort&compress]{natbib}
\usepackage{amsmath}
\usepackage{amssymb}
\usepackage{marginnote}
\usepackage{braket}
\usepackage{xcolor}
\usepackage{mathtools}

\newcommand{\lsim}{\raisebox{-0.13cm}{~\shortstack{$<$ \\[-0.07cm] $\sim$}}~}

\def\beq{\begin{equation}}
\def\eeq{\end{equation}}
\def\beqa{\begin{eqnarray}}
\def\eeqa{\end{eqnarray}}
\def\nn{\nonumber}

\def\ss{\scriptscriptstyle}
\newcommand\tr{\mathop{\mathrm{Tr}}}
\newcommand{\AD}[1]{$\ol{\mbox{D~\,}}\!\!\!$#1}

\newcommand{\avg}[1]{\left\langle#1\right\rangle}

\def\qq{\langle \bar q q \rangle}
%
%\documentclass[12pt]{JHEP3}
%\pdfoutput=1
%
%\renewcommand\floatpagefraction{.001}
%\makeatletter
%\setlength\@fpsep{\textheight}
%\makeatother

%\setlength{\skip\footins}{15pt plus 4pt minus 2 pt}

% Turn off JHEP warnings
%\makeatletter
%\renewcommand\@ENVwarn[1]{}
%\makeatother

\begin{document}

\title{\textbf{ Nonequilibrium chiral fluid dynamics at strong magnetic field }}


\author{Gast\~ao Krein} \email{email}

\author{Carlisson Miller} \email{cmc.pereira@unesp.br}
\affiliation{Instituto de F\'{\i}sica Te\'orica, Universidade Estadual
  Paulista,  Rua Dr. Bento Teobaldo Ferraz, 271 - Bloco II,
  01140-070 S\~ao Paulo, SP, Brazil}

\begin{abstract}
We study the out of equilibrium dynamics of the chiral model in the presence of strong magnetic
 field.
\end{abstract}

\maketitle

%%%%%%%%%%%%%%%%%%%%%%%%%%%%%%%%%%%%%%%%%%%%%%%%%%%%%%%%%%%%%%%%%%%%%%%%%%%%%%%%%%%%%%%%%%%
\section{Introduction}
\label{sec:intro}

The understanding of quantum chromodynamics (QCD) phase structure is one of the important issues of modern physics and it is among the most important goals
of experiments involving heavy ion collisions. Experiments running at Relativistic Heavy Ion Collider (RHIC) and
 Large Hadron Collider (LHC) are expected to probe many questions on QCD phases~\cite{Braun-Munzinger:2015hba,Luo:2017faz,Bzdak:2019pkr}. At high temperatures, and small chemical potential, the lattice
investigations have indicated that the transformation from quark-gluon plasma to hadrons is a crossover transition~\cite{Aoki:2006br,Bazavov:2011nk,Borsanyi:2013hza,Bazavov:2014pvz,Bazavov:2017tot,Borsanyi:2013bia}. On the other hand, for
large chemical potential the lattice has suffered with the sign problem and effective theories have suggested that the phase transition is a first order one~\cite{Stephanov:2007fk,Asakawa:1989bq,Fukushima:2008wg,Carignano:2010ac,Bratovic:2012qs,Fukushima:2013rx}, signalizing a possible existence of a critical endpoint (CEP) on the first-order phase
transition line in the $\mu -T$ plane of the QCD phase diagram.

The interest on the QCD critical point is because order parameter and fluctuations of conserved quantities diverge for the second order phase transition. Such striking feature should be seen in measured in event-by-event fluctuations of particle multiplicities~\cite{Hatta:2002sj,Stephanov:2008qz}. The dynamics of the expansion in a heavy-ion collision is very fast, whereas the relaxation time near to the critical point become large.  This fact leads to the phenomenon of critical slowing down where the system is driven out of equilibrium as it cools through the phase transition. Therefore, due to the fast dynamics of the expanding strongly interacting matter, the critical fluctuations do not have enough time to thermodynamic equilibrium~\cite{Berdnikov:1999ph}. However, it should highlight that the conjectured first-order phase transition derived in effective quark models is only based on the equilibrium thermodynamics. They neglected the nonequilibrium effects during the dynamics of created matter, which could play an important role for a quantitative description of the critical fluctuations~\cite{Asakawa:2016dpt,Mukherjee:2015swa}. The study the dynamics of critical fluctuations properly involves the inclusion of their evolution equations coupled to a fluid dynamical evolution and, in general, involves non-linearities terms.

The strong magnetic field plays an important role in heavy ion collision experiment, affecting directly observables in
collisions at RHIC and LHC~\cite{Shovkovy:2012zn,Fukushima:2012vr,Mueller:2014tea}. Such magnetic field is generated by the spectators (protons) in the collision, and may be captured if the medium produced has finite electric conductivity~\cite{Gursoy:2014aka}. Depending on the centralities of the collisions, the strength of the magnetic can reach $eB \approx 50 m_{\pi}^{2}$~\cite{Kharzeev:2007jp,Skokov:2009qp}, which is a very intense magnetic background compared to the QCD scales($\Lambda^{2}_{QCD} \ll eB$). Probably, biggest magnetic field might have existed in the early Universe as an origin of the present large scale cosmic magnetic field~\cite{Vachaspati:1991nm,Grasso:2000wj}. The strength of this field decreases during the evolution of the formed matter and such time is estimated

The presence of the magnetic field also affects the estimations of the possible effects of the 
deconfinement and chiral phase transitions in heavy ion collisions.  


%%%%%%%%%%%%%%%%%%%%%%%%%%%%%%%%%%%%%%%%%%%%%%%%%%%%%%%%%%%%%%%%%%%%%%%%%%%%%%%%%%%%%%%%%%%

\section{The Linear $\sigma$ model at finite magnetic field }
\label{sec:linear_model}

In this section, we give a brief overview of the linear sigma model which describes the chiral phase transition, where
the $\sigma,~\pi$ field couples to quarks. The Lagrangian reads
%
\begin{eqnarray}
\mathcal{L} &=& \bar{q}[i\gamma^{\mu}\partial_{\mu} - g(\sigma + i\gamma_5 \vec\tau \cdot \vec\pi ) ] q
+\frac{1}{2}\partial_{\mu}\sigma\partial^{\mu}\sigma \nonumber \\
&+& \frac{1}{2}\partial_{\mu}\vec\pi\cdot\partial^{\mu}\vec\pi - U (\sigma,\vec\pi) ,
\label{mathcalL}
\end{eqnarray}
%
where $q = (u, d )$ is the constituent quark field and $g$ is the strength of the coupling between the
quarks and the chiral fields. The interaction between the chiral fields
is given by the potential
%
\begin{equation}
U (\sigma,\vec\pi) = \frac{\lambda}{4} (\sigma^2 + \vec\pi^2 -v^2)^2 - h_{q}\sigma - U_{0},
\label{U}
\end{equation}
%%
where if $h_q$ vanishes the Lagrangian is invariant under $SU(2)_L\times SU(2)_R$. The parameters in~(\ref{U}) are chosen such that chiral symmetry is spontaneously broken in the vacuum, where $\langle \sigma \rangle =f_{\pi} = 93 MeV$  and $\langle\pi\rangle=0$. This way, we use the values for such parameters given in Ref.~\cite{Nahrgang:2011mg}. It is important to highlight that the Lagrangian (\ref{mathcalL}) treats the quarks and antiquarks and the mesons on equal footing. But, in the real world confining forces recombine quarks and antiquarks to form mesons and baryons below the confinement critical temperature. Such aspect is missing in the linear sigma model with constituent quarks.

As we are only interested in the behavior of the order parameters, we neglect fluctuations of the pionic degrees of freedom and keep their values fixed at the vanishing expectation value $\vec{\pi}=\langle \vec\pi \rangle = 0$.

The thermodynamics of the system in the presence of an external magnetic field can be studied in the mean-field approximation. This way, the thermodynamic potential becomes~\cite{Menezes:2008qt}
%
\begin{equation}
\Omega (\sigma, T, \mu) = U (\sigma) + \Omega_{vac} + \Omega_{mag} + \Omega_{med}  ,
\label{Omega}
\end{equation}
%
where one has separated the divergent vacuum contribution from the finite magnetic field contribution so that the regularization method is magnetic field independent. Therefore, we have
%
\begin{equation}
\Omega_{vac} = -2 N_c N_f \int \frac{d^3 p}{(2\pi)^3} E_p ,
\label{Omega_vac}
\end{equation}
%
with $E_p =\sqrt{p^2 + m^2}$, the magnetic field term is
%
\begin{eqnarray}
\Omega_{mag} &=& -N_c \sum_{f=u}^{d} \frac{(\vert q_f\vert B)^2}{2\pi^2} \Big[ \zeta'(-1, x_f )
+ \frac{x_{f}^{2}}{4} \\
&-& \frac{1}{2} \left(x_{f}^{2} -x_{f} \right) \ln(x_f) \Big] ,
\label{Omega_mag}
\end{eqnarray}
%
where we have used $x_f = m^2 /(2\vert q_f\vert B)$ and $\zeta'(-1, x_f ) = d\zeta(z, x_f )/dz \vert_{z=-1}$ being the
zeta Riemann-Hurwitz. The medium contribution takes the form
%
\begin{eqnarray}
\Omega_{med} &=& -\frac{N_c}{2\pi} 2T \sum_{f=u}^{d} \sum_{k=0}^{\infty} \alpha_k (\vert q_f\vert B)  \\
& \times & \int_{0}^{\infty} \frac{dp_z}{2\pi} \ln \left[ 1 + \exp{\left(-\frac{E_k}{T}\right)} \right]
\label{Omega_med}
\end{eqnarray}
%
where $q_u =2e/3$ and $q_d =-e/3$ are the electric charges of the quarks and $m = g \sigma$ the constituent mass. The Landau levels are indexed by $k$ with energies $E_k = \sqrt{p_z^2 + 2k\vert q_f \vert B + m^2}$ and the degeneracy factor $\alpha_k = 2-\delta_{0k}$. Note that although the contributions coming from the flavors $u$ and $d$ in the presence of a magnetic field are different because to their different electric charges, their constituent masses are equal to each other since we
work in isospin-symmetry limit.

It is known that lattice QCD shows a crossover chiral transition, for $\mu = eB =0$, at $T\sim 160$ MeV~\cite{Aoki:2006br}. On the
other hand, some studies claim that at larger densities and lower temperatures, the phase transition becomes of first order and then
ends at a critical point. This way, the authors in Ref.~\cite{Nahrgang:2011mg} tuned the strength of the phase transition by changing the coupling constant $g$. For the coupling $g=3.5$ the present model is able to reproduce usual crossover transition.

In Fig.~\ref{fig1}, we show the thermodynamical effective potential in the presence of magnetic field and analyze the how the order parameter moves with temperature for some high values of magnetic field. Note that the presence of the strong magnetic field does not changes the nature of the chiral transition, it remains crossover. This is not the case when we consider the charged pion loops in the problem, where the chiral transition becomes the first order~\cite{Fraga:2008qn}.
%
\begin{figure}[h]
\begin{center}
\includegraphics[width=9.5cm]{fig1.pdf}
\end{center}
\caption{Thermodynamical potential for high values of magnetic field, with $g=3.5$. The full lines correspond
to that curves with $T=0.1$ GeV and the dashed lines correspond to $T=0.2$ MeV. }
\label{fig1}
\end{figure}
%

Independently of value of $B$, we always have only one minimum which represents the ground state of the system and
 moves to values near to zero as temperature grows. The effect of the strong magnetic field on the chiral dynamics is illustrated
 in Fig.~\ref{fig2} where one note that notable effect of magnetic catalysis: the chiral dynamic is enhanced by the presence of the magnetic field~\cite{Menezes:2008qt}.
%
\begin{figure}[h]
\begin{center}
\includegraphics[width=9.5cm]{fig2.pdf}
\end{center}
\caption{Chiral order parameter as function of temperature for some values of magnetic field, with $g=3.5$ }
\label{fig2}
\end{figure}
%

Another quantity that will be relevant in our description is the thermal mass of the sigma field in the presence of the strong magnetic
field. In the present model, it is obtained as the curvature of the thermodynamic potential at the equilibrium values of the chiral field, as follows
%
\begin{eqnarray}
m_{\sigma}^{2} = \frac{\partial^2 \Omega}{\partial\sigma^2} \Big\vert_{\sigma_{eq}} .
\label{m_sigma}
\end{eqnarray}
%
In Fig.~\ref{fig5}, we show how the behavior of the thermal mass of sigma is affected by the strong magnetic field for $g=3.5$.
%
\begin{figure}[h]
\begin{center}
\includegraphics[width=9.5cm]{fig5.pdf}
\end{center}
\caption{Mass of the sigma field as function of temperature for some values of magnetic field, with $g=3.5$ }
\label{fig5}
\end{figure}
%

%%%%%%%%%%%%%%%%%%%%%%%%%%%%%%%%%%%%%%%%%%%%%%%%%%%%%%%%%%%%%%%%%%%%%%%%%%%%%%%%%%%%%%%%%%%

\section{2PI Effective action formalism}
\label{sec:2PI}
%
In order to study the nonequilibrium dynamics of the system, we employ the known 2PI effective action formalism, which is derived from a functional integral representation of the quantum many-body system. From this action, we can obtain self-consistent quantum equations for the mean field as well as the two-point correlation functions~\cite{Berges:2001fi}. Such formalism is able to take into account dissipation and noise terms, where the propagator is also treated as a dynamical variable. In the one loop approximation, the 2PI action is a functional of the mean field and the propagator and, hence, takes the form
%
\begin{equation}
\Gamma [\sigma, S] = S_{cl}[\sigma] -i \mathrm{Tr}\ln S^{-1} - i\mathrm{Tr}\Sigma S + \Gamma_{2} [\sigma,S],
\label{Gamma}
\end{equation}
%
where the trace is defined as $\mathrm{Tr} = \int_\mathcal{C} dx^4 \sum_{fl} \sum_{Dirac}$ and $\Gamma_{2} [\sigma,S]$ captures the rest which we have not included in the one loop approximation. The integration is realized over the contour $\mathcal{C}$, which is the Keldysh-Schwinger contour. The $S^{ab}$ and $\Sigma^{ab}$ are the full quark propagator and the proper self-energy of the quarks, where the $a$ and $b$ represent indices defined in the Keldysh-Schwinger contour. Other indices are omitted for simplicity.

Following the discussion done in~\cite{Nahrgang:2011mg}, the self energy reads as
%
\begin{equation}
\Sigma^{ab} (x,y) = -i g \delta^{ab}_{\mathcal{C}}(x-y) \sigma^{b}(x),
\label{Sigma}
\end{equation}
%
and, as a result, the third term in Eq.~(\ref{Gamma}) cancels the forth term. Hence, the 2PI effective action simplifies
%
\begin{equation}
\Gamma [\sigma, S] = S_{cl}[\sigma] + i \mathrm{Tr}\ln S .
\label{Gamma2}
\end{equation}
%
We can manipulate the above action by making use of the Schwinger-Dyson equation for quark propagator, at
the chiral limit, written as
%
\begin{equation}
 i\gamma^{\mu}\partial_{\mu} S^{ab} (x,y) - g \sigma^{a} (x) S^{ab} (x,y) =
 i \delta_{\mathcal{C}}^{ab} (x-y) ,
\label{Schwinger-Dyson}
\end{equation}
%
where the quark propagator is affected by mean field $\sigma$. In general, it is quite hard solve such equation because of the space-time dependence of $\sigma^a$. As the 2PI action is not just a function of the sigma field, but also a function of the propagator, we decompose the sigma field into one component $\sigma^{a}_{0}$ that varies slowly and a small fluctuation part $\delta \sigma^a$ as
%
\begin{equation}
\sigma^a (x) = \sigma^{a}_{0} + \delta\sigma^{a}(x) ,
\label{sigmadecomp}
\end{equation}
%
and also we expand the full propagator around the thermal-magnetic propagator, as
%
\begin{equation}
S^{ab}(x,y) = S_{th, B}^{ab}(x,y) + \delta S^{ab}(x,y) + \delta^2 S^{ab}(x,y) ,
\label{propagator_decomp}
\end{equation}
%
where the thermal propagator is that with mass of the quarks generated dynamically $m = g\sigma_0$ and the fluctuations are written in terms of the thermal propagator, as discussed in Ref.~\cite{Nahrgang:2011mg}. Plugging the expressions~(\ref{sigmadecomp}) and (\ref{propagator_decomp}) into the 2PI effective action and expanding in orders of fluctuation, we find
%
\begin{eqnarray}
&&\Gamma [\sigma, S] = S_{cl}[\sigma ] + g \int_{\mathcal{C}}d^{4}x \mathrm{tr}
\left[\delta\sigma^a(x) S_{th, B}^{aa}(x,x)\right]  \nonumber  \\
&-& i \frac{g^2}{2} \int_{\mathcal{C}} d^{4}x d^{4}y  \mathrm{tr} \left[ \delta\sigma^{a}(x) S_{th, B}^{ab}(x,y)
\delta\sigma^{b}(y) S_{th, B}^{ba} (y,x) \right] , \nonumber  \\
&&
\label{Gamma3}
\end{eqnarray}
%
where the $\mathrm{tr}$ is just trace over the flavors and Dirac indices.

Now, we can rewrite explicitly the 2PI action in the closed real time contour and, in order to do so, we write each above term in terms of indices in the contour, $a$ and $b$. Such indices follows the convention: $+$ when the time coordinate finds in $\mathcal{C}_{+}$ and $-$ when the time coordinate finds in $\mathcal{C}_{-}$, see~\cite{Fu:2010ej}. Following~\cite{Nahrgang:2011mg}, we use the center and relative variables
%
\begin{equation}
\delta\bar{\sigma} = \frac{1}{2} (\delta\sigma^{+} + \delta\sigma^{-}), \quad
\Delta\sigma = \delta\sigma^{+} - \delta\sigma^{-} ,
\label{centralcoord}
\end{equation}
%
which are resulted of the combination of the fields on the two time branches of the Keldysh contour. This way, 2PI action becomes
%
\begin{eqnarray}
&&\Gamma [\sigma, S] =  S_{cl}[\sigma] + \frac{g}{2} \int d^4 x \mathrm{tr}\Big[ S_{th, B}^{++}(x,x) \nonumber \\
&+& S_{th, B}^{--}(x,x) \Big]
\Delta\sigma(x) \nonumber \\
&+& i g^2 \int d^4 x d^4 y \theta(x^0 -y^0) \Delta\sigma(x) \mathrm{tr}
\Big[S_{th, B}^{<}(x,y) S_{th, B}^{>} (y,x) \nonumber \\
&-& S_{th, B}^{>} (x,y) S_{th, B}^{<} (y,x) \Big] \delta\bar\sigma(y)  \nonumber \\
&-& i \frac{g^2}{4} \int d^4 x d^4 y \Delta\sigma(x) \mathrm{tr}
\Big[ S_{th, B}^{<} (x,y) S_{th, B}^{>} (y,x) \nonumber \\
&+& S_{th, B}^{>}(x,y) S_{th, B}^{<} (y,x) \Big] \Delta \sigma(y) ,
\label{Gamma4}
\end{eqnarray}
%
where $S^{<}=S^{+-}$ and $S^{>}=S^{-+}$. In Ref.~\cite{Nahrgang:2011mg}, the authors have
 identified the two last terms of the above expression the damping and noise kernels by using
 the influence functional approach. Hence, we rewrite it as
%
\begin{eqnarray}
&&\Gamma [\sigma, S] = S_{cl}[\sigma] + g \int d^4 x 
\mathrm{tr}\Big[ S_{th, B}^{++}(x,x) + S_{th, B}^{--}(x,x) \Big]
 \Delta\sigma(x)\nonumber \\
&+& \int d^4 x D(x) \Delta\sigma(x)
- \frac{i}{2} \int d^4 x d^4 y  \Delta\sigma(x) \mathcal{N}(x,y) \Delta\sigma(y) .\nonumber \\
\label{Gamma5}
\end{eqnarray}
%

To find the equation of motion for the $\sigma$ mean field, we vary the action with respect to $\Delta\sigma$ and we have
%
\begin{eqnarray}
\frac{\delta S_{cl}}{\delta\sigma}
+ g \mathrm{tr} S_{th, B}^{++}(x,x) + D_{\sigma}(x) + \xi_{\sigma}(x) = 0 ,
\label{EoM}
\end{eqnarray}
%
where the sthocastic field $\xi_{\sigma}(x)$ was introduced, in the last term, in order to obtain a real
action, see~\cite{Nahrgang:2011mg}. So, the sthocastic field is determined by the vanishing expectation value
and by the variance, as follows
%
\begin{eqnarray}
\langle \xi_{\sigma}(x) \rangle = 0 , \quad
\langle \xi_{\sigma}(x) \xi_{\sigma}(y) \rangle = \mathcal{N}(x,y) .
\end{eqnarray}
%
The damping kernel is written as
%
\begin{eqnarray}
D_{\sigma}(x) = i g^2 \int d^4 y \theta(x^0 -y^0) M(x-y)\delta\bar\sigma (y) ,
\label{DampingKernel}
\end{eqnarray}
%
with the quantity $M(x-y)$ defined as
%
\begin{eqnarray}
M(x-y) &=& \mathrm{tr}[S_{th, B}^{<}(x-y)S_{th, B}^{>}(x-y) \nonumber  \\
&-& S_{th, B}^{>}(x-y)S_{th, B}^{<}(x-y)] ,
\label{M_quantity}
\end{eqnarray}
%
and the noise kernel, which is the correlation of the sthocastic fields, as
%
\begin{eqnarray}
\mathcal{N}(x,y) &=& - \frac{g^2}{2} \mathrm{tr}
\Big[ S_{th, B}^{<}(x,y) S_{th, B}^{>}(y,x) \nonumber  \\
&+& S_{th, B}^{>}(x,y) S_{th, B}^{<}(y,x) \Big] .
\label{NoiseKernel}
\end{eqnarray}
%
Note that damping and noise contributions arise as second order effects, $\mathcal{O}(g^2)$, in the sthocastic equation for the $\sigma$ field.
%

%
\section{Magnetized quark propagator at real time}
\label{sec:MagnetizedQuark}
%
In order to introduce the magnetic field in the nonequilibrium kernels we use the proper-time method, formulated by Schwinger, to express the magnetized quark propagator~\cite{Schwinger:1951nm}. Such magnetic field
can be generated not only by the a unique vector potential and, for simplicity, we assume that the magnetic field $B$ is constant and homogeneous in the $z-$direction. This way, the quark
propagator can be written as
%
\begin{equation}
S (x,x') = \phi(x,x') \int\frac{d^4 p}{(2\pi)^4} e^{ip(x-x')} S(p) ,
\label{propagator_magnetic}
\end{equation}
%
where the phase factor $\phi(x,x')$ is given by
%
\begin{equation}
\phi(x,x') = \exp\left( ie \int_{x}^{x'}  d\xi A(\xi) \right) ,
\label{phase_factor}
\end{equation}
%
which becomes $\phi(x,x') = 1$ for closed fermion loop with two lines~\cite{Chyi:1999fc}. For the constant magnetic field, the propagator can
be expanded in terms of the Laguerre polynomials ($L_n$), where the order of the Laguerre polynomial corresponds to energy
eigenvalues of the fermion in the magnetic field, known as Landau levels, see~\cite{Hasan:2017fmf}. At strong magnetic field
the quarks occupy the lowest Landau levels. In this work, we will use lowest Landau level~(LLL) $n=0$ and next-lowest
Landau level(NLLL) $n=1$. This way, the quark propagator acquires three contributions, as follows
%
\begin{equation}
S_{B}(p) = S_{1} (p) + S_{2} (p) + S_{3} (p)
\label{propagator_magnetic1}
\end{equation}
%
where the first term comes from the LLL $n=0$, given by
%
\begin{equation}
S_{1}(p) = \left( 1 + \gamma^0 \gamma^3 \gamma^5 \right) \frac{\gamma \cdot p_{\parallel}}{p_{\parallel}^2 - m^2}
e^{-\frac{p_T^2}{\vert qB\vert}}
\label{S1}
\end{equation}
%
with $m$ being the constituent quark mass and $\gamma \cdot p_{\parallel} =\gamma^0 p^0 -\gamma^3 p^3$. The second last
terms comes from the NLLL $n=1$, are
%
\begin{eqnarray}
S_{2}(p) &=& 2 \left( \alpha - ( 1- \alpha) \gamma^0 \gamma^3 \gamma^5 \right)
 \frac{\gamma \cdot p_{\parallel}}{p_{\parallel}^2 - m_B^2} e^{-\frac{p_T^2}{\vert qB\vert}}\nonumber \\
S_{3}(p) &=& - 2 \frac{\gamma\cdot p_T}{eB} \frac{p_{\parallel}^2 - m^2}{p_{\parallel}^2 - m_B^2}
e^{-\frac{p_T^2}{\vert qB\vert}}
\label{S2S3}
\end{eqnarray}
%
where $\alpha=p_T^2 / \vert qB\vert$ and $\gamma \cdot p_{T} =\gamma^1 p^1 +\gamma^2 p^2$. Note that in the NLLL the
mass quarks, modified by $m_B^2 = m^2 + 2\vert eB\vert$, is enhanced by the presence magnetic. Surely, this will bring
interesting consequences in our results.

The presence of the finite temperature introduces a new scale in the problem which plays with the magnetic field. This way,
the strong magnetic field approximation means $eB \gg T^2$. In the real time, the temperature is introduced in the magnetized propagator by diagonalizing it via the matrix $U(p)$ as~\cite{Mallik:2009pj}
%
\begin{eqnarray}
S_{th,B}^{ab}(p) =
\begin{pmatrix}
S_{B}(p) &  0  \\
0 &  -S_{B}^{\ast}(p)
\end{pmatrix}_{ab}
+ A(p) U_{ab} (p) ,
\label{SthB}
\end{eqnarray}
%
%
%\begin{eqnarray}
%S_{th,B}^{ab}(p) = U_{ac}(p)
%\begin{pmatrix}
%S_{B}(p) &  0  \\
%0 &  -S_{B}^{\ast}(p)
%\end{pmatrix}
%_{cd} U_{db}(p) ,
%\end{eqnarray}
%
where the matrix $U(p)$ is given by
%
\begin{eqnarray}
U (p) =
\begin{pmatrix}
n_p(p^0) & n_p(p^0) - \theta(-p^0)  \\
n_p(p^0) - \theta(p^0) &  n_p(p^0)
\end{pmatrix} ,
\label{UMatrix}
\end{eqnarray}
%
%
%\begin{eqnarray}
%U (p) =
%\begin{pmatrix}
%\sqrt{1 - n_p(p^0)} & - \sqrt{n_p(p^0)}  \\
%\sqrt{n_p(p^0)} & \sqrt{1 - n_p(p^0)}
%\end{pmatrix} ,
%\end{eqnarray}
%
with $n_p(p^0) = 1 / (1 + e^{\vert p^0 \vert / T})$ in equilibrium and the letters $a$ and $b$ representing
indices on the Keldysh contour. The spectral density, evaluated as $A(p) = S_{B}(p) - S_{B}^{\ast}(p)$, is the sum
of three components obtained from the Landau levels, where each component is written as
%
\begin{eqnarray}
A_{1}(p) &=& 2\pi i \left( 1 + \gamma^0 \gamma^3 \gamma^5 \right) \left( \gamma \cdot p_{\parallel} + m \right)
e^{-\frac{p_T^2}{\vert qB\vert}} \delta\left( p_0^2 - E_z^2 \right) \nonumber  \\
A_{2}(p) &=& 4\pi i \left( \alpha - (1-\alpha) \gamma^0 \gamma^3 \gamma^5 \right) \left( \gamma \cdot p_{\parallel} + m \right)
e^{-\frac{p_T^2}{\vert qB\vert}} \nonumber  \\
&\times& \delta\left( p_0^2 - (E_z^B)^2 \right) \nonumber  \\
A_{3}(p) &=& - 4\pi i \frac{\gamma \cdot p_{T}}{eB} \left( p_{\parallel}^{2 } - m^2 \right)
e^{-\frac{p_T^2}{\vert qB\vert}} \delta\left( p_0^2 - (E_z^B)^2 \right) .
\label{Aspectral}
\end{eqnarray}
%
Therefore, we conclude that each component of the magnetized quark propagator on the real-time contour, $S_{th,B}^{ab}(p)$, acquires three contributions because of the Landau levels.

%%%%%%% %%%%%%%%%%%%%%%%%%%%%%%%%%%%%%%%%%%%%%%%%%%%%%%%%%%%%%%%%%%%%%%%%%%%%%%%%%%%%%%%%%%%
\section{Equation of motion for the $\sigma$ field}
\label{sec:EoM}

In this section, we use explicitly evaluate each term of the equation of motion for the $\sigma$
field~(\ref{EoM}). For this task, we remind that the free fermionic propagators on the real-time contour
obeys the relation~\cite{Nahrgang:2011mg}
%
\begin{eqnarray}
S_{th,B}^{++}(x,y) = S_{th,B}^{>}(x,y)\theta(x^0 - y^0) + S_{th,B}^{<}(x,y)\theta(y^0 - x^0) \nonumber \\
S_{th,B}^{--}(x,y) = S_{th,B}^{<}(x,y)\theta(x^0 - y^0) + S_{th,B}^{>}(x,y)\theta(y^0 - x^0) \nonumber . \\
\label{SppSmm}
\end{eqnarray}
%

Other relations between propagators on the the real-time Keldysh contour can be obtained, see~\cite{Fu:2010ej}.

\subsection{Lowest order}
%\label{sec:EoMfield}
We mean by lowest term that which is linear in the coupling constant in Eq.~(\ref{EoM}). This way,
using the relations~(\ref{SppSmm}), we obtain
%
\begin{eqnarray}
g \mathrm{tr} S_{th,B}^{++}(x,x) &=& -\frac{mg}{(4\pi)^2} d_q eB
\int_{0}^{\infty} dp_z \Big[ \frac{1}{E_z} \left( 1 - 2n_F (E_z ) \right) \nonumber \\
&+& \frac{2}{E_z^B} \left( 1 - 2n_F (E_z^B ) \right)  \Big] , \nonumber \\
g \mathrm{tr} S_{th,B}^{++}(x,x) &=& - g \rho_s
\label{rho}
\end{eqnarray}
%
where the energies are $E_z = \sqrt{p_z^2 + m^2}$ and $E_z^B = \sqrt{p_z^2 + m_B^2}$ and we also have used the
one-loop scalar density $\rho_s $ to rewrite our expression. The first contribution comes from the
$n=0$ Landau level and the second one from the first contribution of the $n=1$ Landau level, the second contribution
vanishes due to only Dirac matrix.

\subsection{Damping kernel}
\label{sec:Damping kernel}
%
Following the considerations discussed in Ref.~\cite{Nahrgang:2011mg}, we use the harmonic initial oscillations and, hence, one can rewrite the damping kernel~(\ref{DampingKernel}) in momentum space as
%
\begin{eqnarray}
D_{\sigma}(x) = - g^2 \int \frac{d^3 k}{(2\pi)^3} \frac{1}{2E_k} e^{-i \vec{k} \cdot \vec{x} }
M(E_k, \vec{k}) \partial_{t}\bar\sigma (t,\vec{k}) , \nonumber \\
\label{DampingKernel1}
\end{eqnarray}
%
where $M(E_k, \vec{k})$ contains on-shell reaction rate of the processes that lead to the dissipative contribution. Being interested in the long-range oscillations of $\sigma$, we calculate the damping kernel for the zero mode, $\vec{k} = \vec{0}$, of the $\sigma$ mean field and approximate $M(E_k, \vec{k}) \approx M(m_{\sigma} , \vec{0})$, obtaining
%
\begin{eqnarray}
D_{\sigma}(x) &\approx& - g^2 \int \frac{d^3 k}{(2\pi)^3} \frac{1}{2m_{\sigma}} e^{-i \vec{k} \cdot \vec{x} }
 M(m_{\sigma},0) \partial_{t}\bar\sigma (t,\vec{k}) \nonumber  \\
&=& - \eta \partial_{t}\bar\sigma (t,x) ,
\label{DampingKernel2}
\end{eqnarray}
%
where we have identified the damping coefficient $\eta$ as
%
\begin{eqnarray}
\eta = \frac{g^2}{2m_{\sigma}} M( m_{\sigma},\vec{0} ).
\label{eta}
\end{eqnarray}
%
This way, finding the complete thermo-magnetic dependence of the damping coefficient requires evaluate explicitly
the momentum space of the quantity $M(x-y)$. Hence, once that each real-time propagators have three contributions due to the Landau levels, we can write the condensate form $M(k) = \sum_{c,c'} M_{cc'}(k)$, where
%
\begin{eqnarray}
M_{cc'}(k) &=& \int \frac{d^4 p}{(2\pi)^4} \mathrm{tr}[S_{th, B}^{c,<}(k+p)S_{th, B}^{c',>}(p) \nonumber  \\
&-& S_{th, B}^{c,>}(k+p)S_{th, B}^{c',<}(p)] ,
\label{M_quantity1}
\end{eqnarray}
%
where the indices are $c,~c'=1$ corresponds to LLL and $c,~c'=2,3$ corresponds to NLLL. This way, using the
expressions in~(\ref{SthB}), we can rewrite the above quantity as
%
\begin{eqnarray}
M_{cc'}(k) &=& \int \frac{d^4 p}{(2\pi)^2} e^{-\frac{(k_T + p_T)^2}{eB}} e^{-\frac{p_T^2}{eB}}
\mathrm{tr}[\hat{A}_{c}(k+p)\hat{A}_{c'}(p)] \nonumber  \\
&\times& D(k^0 + p^0 ,p^0) \delta\left[(k_0 +p_0)^2 - (E_z^c)^2 \right] \nonumber  \\
&\times& \delta\left[p_0^2 - (E_z^{c'})^2 \right] ,
\label{M_quantity2}
\end{eqnarray}
%
where we have used the spectral densities, Eqs.~(\ref{Aspectral}), and inserted all the Dirac matrices, contained in them, in $\hat{A}_{c}(p)$ function. Besides, the $D(k^0 + p^0 ,p^0)$ function involves the Fermi distribution, as follows
%
\begin{eqnarray}
D(k^0 + p^0 ,p^0) &=& \left[ n_F (k^0 + p^0) - \theta(-k^0 - p^0) \right] \nonumber  \\
&\times& \left[ n_F (p^0) - \theta(p^0) \right] \nonumber  \\
&-& \left[ n_F (k^0 + p^0) - \theta(k^0 + p^0) \right] \nonumber  \\
&\times& \left[ n_F (p^0) - \theta(-p^0) \right] .
\label{Dfunction}
\end{eqnarray}
%
The indices in Eq.~(\ref{M_quantity2}) show that we have to evaluate nine components. The indexed energies follow the notation $(E_z^1)^2 = (E_z)^2$ and $(E_z^2)^2 = (E_z^3)^2 = (E_z^B)^2$. By using the expressions for the spectral densities~(\ref{Aspectral}), we show that those components with $c\neq c'$ are null due to the trace over the Dirac matrices, eliminating six components. So, we have left only components with $c = c'$.

This way, evaluating all trace for the cases $c = c'$, we obtain these three components
%
\begin{eqnarray}
\mathrm{tr}[\hat{A}_{1}(k+p)\hat{A}_{1}(p)] &=& 8 \left[ p^0 \left(k^0 +p^0 \right) - p^3 \left(k^3 +p^3 \right) + m^2 \right] , \nonumber  \\
\mathrm{tr}[\hat{A}_{2}(k+p)\hat{A}_{2}(p)] &=& 16 \left[ 1- \alpha^2 + \alpha^4 \right] \mathrm{tr}[A_{1}(k+p)A_{1}(p)], \nonumber  \\
\mathrm{tr}[\hat{A}_{3}(k+p)\hat{A}_{3}(p)] &=& - \frac{\left[ (p_{\parallel} + k_{\parallel})^2 - m^2 \right]^2}{(eB)^2}
\Big[ p^1 (k^1 +p^1 ) \nonumber  \\
&+& p^2 (k^2 +p^2 ) \Big]
\label{Trace1}
\end{eqnarray}
%
where one note that transversal momentum do not mix with the other components. This allows us to perform the calculation of the quantity $M_{cc'}(k)$ by separating the transversal part from the longitudinal one and applying the integration over the delta functions.

As we have considered long-range oscillations of the $\sigma$ mean field with zero
mode, $\vec{k} = 0$, to evaluate the damping kernel, we can write $M_{cc}(E_k, \vec{k}) \approx M_{cc}(m_{\sigma}, \vec{0})$. This way, we find for the lowest Landau level (LLL)
%
\begin{eqnarray}
M_{11}(m_{\sigma}, \vec{0}) = \frac{1}{2\pi} \sqrt{m_{\sigma}^2 -4m^2} \frac{eB}{m_{\sigma}}
\left[ 1 - 2 n_F \left(\frac{m_{\sigma}}{2}\right) \right] , \nonumber  \\
\label{M_11}
\end{eqnarray}
%
where one can see that the only process which obeys $m_{\sigma} > 2 m$ are kinematically allowed. This
case follows the same condition for the case when the magnetic field is absent, as showed in Ref.~\cite{Nahrgang:2011mg}. For the next lowest Landau levels (NLLL), following the same algebraical
procedure, we find
%
\begin{eqnarray}
M_{22}(m_{\sigma}, \vec{0}) &=& \frac{2}{\pi} \frac{m_{\sigma}^2 -(m^2 +eB)}{\sqrt{m_{\sigma}^2 -4m_B^2}} \frac{eB}{m_{\sigma}}
\left[ 1 - 2 n_F \left(\frac{m_{\sigma}}{2}\right) \right] , \nonumber  \\
M_{33}(m_{\sigma}, \vec{0}) &=& \frac{1}{\pi} \frac{1}{\sqrt{m_{\sigma}^2 -4m_B^2}} \frac{(eB)^2}{m_{\sigma}}
\left[ 1 - 2 n_F \left(\frac{m_{\sigma}}{2}\right) \right], \nonumber  \\
\label{M_22_33}
\end{eqnarray}
%
where in such case, due to the fact that the magnetic field enhances the quark mass, the NLLL only plays role for the process where $m_{\sigma} > 2 m_B$. Besides, we point out that the part involving the Fermi distribution
has the same structure from that obtained when the magnetic field is absent. This is expected once that the $U(p)$ matrix~(\ref{UMatrix}), which incorporates the Fermi distribution, is not affected by the magnetic field.

Therefore, we have three contributions for the damping kernel and, as a consequence, we can write each contribution for the damping coefficient: for the lowest Landau level, one write
%
\begin{eqnarray}
\eta_{11} = \frac{g^2}{2\pi} \sqrt{m_{\sigma}^2 -4m^2} \frac{eB}{2m_{\sigma}^2}
\left[ 1 - 2 n_F \left(\frac{m_{\sigma}}{2}\right) \right],
\label{eta11}
\end{eqnarray}
%
and for the next lowest Landau level, we have
%
\begin{eqnarray}
\eta_{22} &=& \frac{2g^2}{\pi} \frac{m_{\sigma}^2 -(m^2 +eB)}{\sqrt{m_{\sigma}^2 -4m_B^2}}
\frac{eB}{2m_{\sigma}^2} \left[ 1 - 2 n_F \left(\frac{m_{\sigma}}{2}\right) \right] , \nonumber  \\
\eta_{33} &=& \frac{g^2}{\pi} \frac{1}{\sqrt{m_{\sigma}^2 -4m_B^2}} \frac{(eB)^2}{2m_{\sigma}^2}
\left[ 1 - 2 n_F \left(\frac{m_{\sigma}}{2}\right) \right].
\label{eta_22_33}
\end{eqnarray}
%

Using the results for the constituent quark mass and sigma mass, obtained in section~(\ref{sec:linear_model}), and working with strong magnetic fields with values $eB/m_{\pi}^2 = 5, 15, 20$, we conclude that the magnetic quark mass is large enough to satisfy the condition $m_{\sigma} > 2 m_B$. Then, such processes are not allowed. This way, the effect of the next Landau level, $n=1$, plays no role on the damping coefficient, making $\eta_{22}=\eta_{33}=0$.

In Fig.~\ref{fig7}, we show the temperature dependence of the damping coefficient for $g = 3.5$ for three values of the magnetic fields. The effect of the strong magnetic field increases the temperature where the damping coefficient becomes non-null: this is a direct consequence of the magnetic catalysis present in the model.
%
\begin{figure}[h]
\begin{center}
\includegraphics[width=9.5cm]{fig7.pdf}
\end{center}
\caption{Temperature dependence of damping coefficient for strong magnetic field with $g=3.5$ }
\label{fig7}
\end{figure}
%

One can also check that the magnetic field increases the speed of growth of the damping coefficient with the
temperature.
%

\subsection{Correlation of the noise fields}
\label{sec:Noise_kernel}

Now, we verify the effect of the strong magnetic field on the correlation of the noise fields at finite
temperature. Following the procedure demonstrated in Ref.~\cite{Nahrgang:2011mg}, variance of the noise fields is moved to the momentum space and, then, by using the approximation for the zero mode, we write
%
\begin{eqnarray}
\langle \xi (\vec{x}, t) \xi (\vec{x}',t') \rangle = \mathcal{N}( m_{\sigma} , \vec{0} )
\delta^3 (x - x') \delta ( t - t') . \quad
\label{noiseforce1}
\end{eqnarray}
%
Therefore, we have determined the variance of the noise fields in momentum space, which is written as
%
\begin{eqnarray}
\mathcal{N}( k ) &=& - \frac{g^2}{2} \int \frac{d^4 p}{(2\pi)^4} \mathrm{tr}
\Big[ S_{th, B}^{<}( k + p ) S_{th, B}^{>}( p ) \nonumber  \\
&+& S_{th, B}^{>}(k + p ) S_{th, B}^{<}( p ) \Big] .
\label{NoiseKernel1}
\end{eqnarray}

By using the three contributions of the quark propagators on the Keldysh contour, discussed in section~(\ref{sec:MagnetizedQuark}), the noise kernel can be also written as $\mathcal{N}(k) = \sum_{c,c'} \mathcal{N}_{cc'}(k)$ where the indices follows same definition of damping kernel. Besides,
one note that the basic difference between the expressions for noise kernel and the damping kernel is the
sum operation in Eq.~(\ref{NoiseKernel1}), hence we follow the same procedure as before and we obtain just three terms, when $c=c'$. This
way, we have
%
\begin{eqnarray}
\mathcal{N}_{cc}(m_{\sigma}, \vec{0}) &=& m_{\sigma} \eta_{cc}
 \coth\left(\frac{m_{\sigma}}{2T} \right) ,
\label{NoiseKernel2}
\end{eqnarray}
%
where we have written it in a general form, making clear the explicit temperature dependence.

Finally, the explicit equation of motion for the $\sigma$ field becomes
%
\begin{eqnarray}
\partial_{\mu}\partial^{\mu} \sigma + \frac{\delta U }{\delta\sigma}
+ g \rho_s + \eta \partial_t \sigma(x) = \xi (x) ,
\label{EoM2}
\end{eqnarray}
%
where the scalar density $\rho_s$ is defined in Eq.~(\ref{rho}), the three contributions of the damping coefficient are given in Eqs.~(\ref{eta11}, \ref{eta_22_33}) and the correlation of the noise field is in
Eq.~(\ref{noiseforce1}).


\section{Solutions of the Langevin equation for long and short times }
\label{sec:Solutions}
%
Firstly, we will investigate the behavior of the solution of the Langevin equation for the broken phase for long times, near to the thermal equilibrium, $\sigma_0$. So, we take the field configuration which varies slowly and, hence, the second derivative in time can be negligible. Then, we write
%
\begin{eqnarray}
\sigma(x) = \sigma_0 + \delta\tilde\sigma(x),
\end{eqnarray}
%
and the Langevin equation~(\ref{EoM2}) becomes
%
\begin{eqnarray}
\eta \frac{\partial \delta\tilde\sigma(x)}{\partial t} - \nabla^2 \delta\tilde\sigma(x) + g \rho_s
+ \lambda \left( 3 \sigma_0 - v^2 \right) \delta\tilde\sigma(x) = \xi (x) .\nonumber  \\
\end{eqnarray}
%
Once that the above equation is linear we can solve it going to 3d Fourier space. Doing so, the general solution of
the equation is
%
\begin{eqnarray}
\delta\tilde\sigma(\vec{k}, t) &=& - \frac{g \rho_s}{k^2 + m_0^2 }
+ \sigma(\vec{k}, 0) e^{ -(k^2 + m_0^2) \frac{t}{\eta} } \nonumber  \\
&+& \frac{ e^{-(k^2 + m_0^2) \frac{t}{\eta} } }{\eta}
\int_0^t \xi (\vec{k}, t') e^{ -(k^2 + m_0^2) \frac{t'}{\eta} } dt'
\end{eqnarray}
%
where we have used $ m_0^2 = \lambda \left( 3 \sigma_0 - v^2 \right)$. With such solution in hands, we can
evaluate an important quantity of interest, the correlation function, whose explicit solution is
%
\begin{eqnarray}
S(k,t) &=& \left[- \frac{g \rho_s}{k^2 + m_0^2} + \sigma(\vec{k}, 0) e^{ -(k^2 + m_0^2) \frac{t}{\eta} }
\right]^2 \nonumber  \\
&+& \frac{N(m_{\sigma}, 0)}{2\eta} \frac{1}{k^2 + m_0^2} \left(1 - e^{ -(k^2 + m_0^2) \frac{t}{\eta}} \right) ,
 \end{eqnarray}
%
where, for large times when the system thermalize, one can write
%
\begin{eqnarray}
S(k,t\rightarrow \infty) \approx \frac{g^2 \rho_s^2}{(k^2 + m_0^2)^2} +
 \frac{\mathcal{N}(m_{\sigma}, 0)}{2\eta(k^2 + m_0^2)}  ,
  \end{eqnarray}
%
which implies that the correlations, near to the equilibrium, decay exponentially in the configuration
space as
%
\begin{eqnarray}
S(r,t\rightarrow \infty) \approx \frac{g^2 \rho_s^2}{2} \frac{e^{-m_0 r}}{r} +
 \frac{\mathcal{N}(m_{\sigma}, 0)}{2\eta}  \frac{e^{-m_0 r}}{r}  ,
  \end{eqnarray}
%
namely, the ``mass'' $m_0$ of the system controls the correlation decay near to the equilibrium. Also note
that, the strength of the correlation is controlled by the scalar density~(\ref{rho}) and by the ratio
$\mathcal{N}(m_{\sigma}, 0) / 2\eta$, in Eq.~(\ref{NoiseKernel2}). In Fig.~\ref{fig10}, we show the correlation function as function of the position for the $T = 0.22$~GeV where one can confirm that the strength of the correlation is enhanced with the increase of the magnetic field.
%
\begin{figure}[h]
\begin{center}
\includegraphics[width=9.5cm]{fig10.pdf}
\end{center}
\caption{damping coefficient for $g=3.5$ }
\label{fig10}
\end{figure}
%

At high temperature limit, the Fig.~\ref{fig5} reveals the linear temperature dependence of the sigma
mass, $m_{\sigma} \sim T$. This simplifies the temperature dependence of the noise kernel~(\ref{NoiseKernel2})
and, therefore, $\mathcal{N}(m_{\sigma}, 0) / 2\eta$  goes linearly with temperature. Besides, one can show
that the scalar density~(\ref{rho}) vanishes, once that the Fermi distribution goes to 1, obtaining
%
\begin{eqnarray}
S(r,t\rightarrow \infty) \approx T  \frac{e^{-m_0 r}}{r} .
\end{eqnarray}
%

Another important limit is the short time, $t\approx 0$, where the interest situation is to study the evolution
of the system when $\sigma(x,t\approx 0 ) \approx 0$. In such case, we can neglect the quartic term in
potential $U(\sigma)$ and the scalar density $\rho_s$. This way, the Langevin equation~(\ref{EoM2}) becomes
%
\begin{eqnarray}
\eta \frac{\partial \sigma(x)}{\partial t} - \nabla^2 \sigma(x) - \lambda v^2 \sigma - h_q = \xi (x) ,
\end{eqnarray}
%
where we have neglected the second derivative in time. Hence, the general solution of the above equation
is obtained similarly to that obtained for long times. So, we have
%
\begin{eqnarray}
\sigma(\vec{k}, t) &=& - \frac{h_q}{k^2 - \lambda v^2 } + \sigma(\vec{k}, 0)
e^{ -(k^2 - \lambda v^2) \frac{t}{\eta} } \nonumber  \\
&+& \frac{ e^{-(k^2 - \lambda v^2) \frac{t}{\eta} } }{\eta}
\int_0^t \xi (\vec{k}, t') e^{ -(k^2 - \lambda v^2) \frac{t'}{\eta} } dt' .
\label{short_time}
\end{eqnarray}
%
We note that, for the case $k^2 < \lambda v^2$, the solutions grows exponentially in time, which implies
that the fluctuations of long wavelength lead to growth of the order parameter for short times. Such
result is well known as spinodal decomposition. In addition, the correlation function obtained from that general solution, Eq.~(\ref{short_time}), oscillates with time. For larger times, the quartic term plays role and it tends to relax the solutions to the equilibrium.

%%%%%%% %%%%%%%%%%%%%%%%%%%%%%%%%%%%%%%%%%%%%%%%%%%%%%%%%%%%%%%%%%%%%%%%%%%%%%%%%%%%%%%%%%%%
\section{Conclusion and final remarks}
\label{sec:conclusion}

In this work, we have employed the 2PI (two-particle irreducible) effective action formalism in order
to study the out-of-equilibrium chiral fluid dynamic in the presence of strong magnetic field. We start
by presenting, in section~\ref{sec:linear_model}, the linear sigma model which is used to describe the
dynamic of the, $\sigma$ field, which is the order parameter of chiral symmetry
breaking. In that model, we implemented temperature and magnetic field, via Matsubara
 formalism, in order to find some equilibrium quantities, as constituent quark mass and the sigma mass. By analyzing the effective potential, we showed that the model reproduces magnetic catalysis for a second order
 transition.

In section~\ref{sec:2PI}, using an approximated form for the self-energy, we explicitly have used the 2PI effective action formalism and found that $\sigma$ field is propagated following a sthocastic second order
differential equation, where the quark-antiquark bath gives rise to the damping term and a stochastic field. Following same steps done in Ref.~\cite{Nahrgang:2011mg}, we found the explicit form for the damping kernel and for the noise kernel.

Using the proper-time Schwinger method, in section~\ref{sec:MagnetizedQuark}, we have introduced a strong magnetic field in the $z$-direction in the real-time thermal quark propagator and, in
section~\ref{sec:EoM}, for the zero mode, we were able to evaluate the effect of the two lowest Landau levels on the damping and noise terms. For $n=0$ case, only process with $m_{\sigma} > 2 m$ are kinematically allowed and those terms scale linearly with $B$. On the other hand, as the magnetic field modifies the quark mass for $m_B$, the Landau level with $n=1$ only contributes for process with $m_{\sigma} > 2 m_B$. So, the damping and noise terms scales up to $B^2$. Our numerical results have shown that the temperature, where the damping coefficient shows up, tends to grow with magnetic field and this is a consequence of the magnetic catalysis. This way, we expect that models which reproduce inverse magnetic catalysis phenomenon, such as NJL model, would produce a finite damping kernel at lower temperatures by increasing the magnetic field. Besides, the increases of the magnetic field makes the damping coefficient growing more rapidly with temperature.


%%%%%%%%%%%%%%%%%%%%%%%%%%%%%%%%%%%%%%%%%%%%%%%%%%%%%%%%%%%%%%%%%%%%%%%%%%%%%%%%%%%%%%%%%%%
\section*{Acknowledgements}

The work of C.M has been supported by the Brazilian Science Foundation CAPES.


%%%%%%%%%%%%%%%%%%%%%%%%%%%%%%%%%%%%%%%%%%%%%%%%%%%%%%%%%%%%%%%%%%%%%%%%%%%%%%%%%%%%%%%%%%%
\section{References}
\begin{thebibliography}{99}

%\cite{Braun-Munzinger:2015hba}
\bibitem{Braun-Munzinger:2015hba}
P.~Braun-Munzinger, V.~Koch, T.~Schäfer and J.~Stachel,
%``Properties of hot and dense matter from relativistic heavy ion collisions,''
Phys. Rept. \textbf{621}, 76-126 (2016)
doi:10.1016/j.physrep.2015.12.003
[arXiv:1510.00442 [nucl-th]].
%114 citations counted in INSPIRE as of 12 May 2020

%\cite{Luo:2017faz}
\bibitem{Luo:2017faz}
X.~Luo and N.~Xu,
%``Search for the QCD Critical Point with Fluctuations of Conserved Quantities in Relativistic Heavy-Ion Collisions at RHIC : An Overview,''
Nucl. Sci. Tech. \textbf{28}, no.8, 112 (2017)
doi:10.1007/s41365-017-0257-0
[arXiv:1701.02105 [nucl-ex]].
%174 citations counted in INSPIRE as of 12 May 2020

%\cite{Bzdak:2019pkr}
\bibitem{Bzdak:2019pkr}
A.~Bzdak, S.~Esumi, V.~Koch, J.~Liao, M.~Stephanov and N.~Xu,
%``Mapping the Phases of Quantum Chromodynamics with Beam Energy Scan,''
Phys. Rept. \textbf{853}, 1-87 (2020)
doi:10.1016/j.physrep.2020.01.005
[arXiv:1906.00936 [nucl-th]].
%50 citations counted in INSPIRE as of 12 May 2020

%\cite{Aoki:2006br}
\bibitem{Aoki:2006br}
  Y.~Aoki, Z.~Fodor, S.~D.~Katz and K.~K.~Szabo,
  %``The QCD transition temperature: Results with physical masses in the continuum limit,''
  Phys.\ Lett.\ B {\bf 643}, 46 (2006)
  doi:10.1016/j.physletb.2006.10.021
  [hep-lat/0609068].
  %%CITATION = doi:10.1016/j.physletb.2006.10.021;%%
  %762 citations counted in INSPIRE as of 04 Feb 2020

%\cite{Bazavov:2011nk}
\bibitem{Bazavov:2011nk}
A.~Bazavov, T.~Bhattacharya, M.~Cheng, C.~DeTar, H.~Ding, S.~Gottlieb, R.~Gupta, P.~Hegde, U.~Heller, F.~Karsch, E.~Laermann, L.~Levkova, S.~Mukherjee, P.~Petreczky, C.~Schmidt, R.~Soltz, W.~Soeldner, R.~Sugar, D.~Toussaint, W.~Unger and P.~Vranas,
%``The chiral and deconfinement aspects of the QCD transition,''
Phys. Rev. D \textbf{85}, 054503 (2012)
doi:10.1103/PhysRevD.85.054503
[arXiv:1111.1710 [hep-lat]].
%932 citations counted in INSPIRE as of 12 May 2020

%\cite{Borsanyi:2013hza}
\bibitem{Borsanyi:2013hza}
S.~Borsanyi, Z.~Fodor, S.~Katz, S.~Krieg, C.~Ratti and K.~Szabo,
%``Freeze-out parameters: lattice meets experiment,''
Phys. Rev. Lett. \textbf{111}, 062005 (2013)
doi:10.1103/PhysRevLett.111.062005
[arXiv:1305.5161 [hep-lat]].
%165 citations counted in INSPIRE as of 12 May 2020

%\cite{Bazavov:2014pvz}
\bibitem{Bazavov:2014pvz}
A.~Bazavov \textit{et al.} [HotQCD],
%``Equation of state in ( 2+1 )-flavor QCD,''
Phys. Rev. D \textbf{90}, 094503 (2014)
doi:10.1103/PhysRevD.90.094503
[arXiv:1407.6387 [hep-lat]].
%704 citations counted in INSPIRE as of 12 May 2020

%\cite{Bazavov:2017tot}
\bibitem{Bazavov:2017tot}
A.~Bazavov \textit{et al.} [HotQCD],
%``Skewness and kurtosis of net baryon-number distributions at small values of the baryon chemical potential,''
Phys. Rev. D \textbf{96}, no.7, 074510 (2017)
doi:10.1103/PhysRevD.96.074510
[arXiv:1708.04897 [hep-lat]].
%46 citations counted in INSPIRE as of 12 May 2020

%\cite{Borsanyi:2013bia}
\bibitem{Borsanyi:2013bia}
S.~Borsanyi, Z.~Fodor, C.~Hoelbling, S.~D.~Katz, S.~Krieg and K.~K.~Szabo,
%``Full result for the QCD equation of state with 2+1 flavors,''
Phys. Lett. B \textbf{730}, 99-104 (2014)
doi:10.1016/j.physletb.2014.01.007
[arXiv:1309.5258 [hep-lat]].
%600 citations counted in INSPIRE as of 12 May 2020

%\cite{Stephanov:2007fk}
\bibitem{Stephanov:2007fk}
M.~Stephanov,
%``QCD phase diagram: An Overview,''
PoS \textbf{LAT2006}, 024 (2006)
doi:10.22323/1.032.0024
[arXiv:hep-lat/0701002 [hep-lat]].
%317 citations counted in INSPIRE as of 13 May 2020

%\cite{Asakawa:1989bq}
\bibitem{Asakawa:1989bq}
M.~Asakawa and K.~Yazaki,
%``Chiral Restoration at Finite Density and Temperature,''
Nucl. Phys. A \textbf{504}, 668-684 (1989)
doi:10.1016/0375-9474(89)90002-X
%566 citations counted in INSPIRE as of 13 May 2020

%\cite{Fukushima:2008wg}
\bibitem{Fukushima:2008wg}
K.~Fukushima,
%``Phase diagrams in the three-flavor Nambu-Jona-Lasinio model with the Polyakov loop,''
Phys. Rev. D \textbf{77}, 114028 (2008)
doi:10.1103/PhysRevD.77.114028
[arXiv:0803.3318 [hep-ph]].
%452 citations counted in INSPIRE as of 13 May 2020

%\cite{Carignano:2010ac}
\bibitem{Carignano:2010ac}
S.~Carignano, D.~Nickel and M.~Buballa,
%``Influence of vector interaction and Polyakov loop dynamics on inhomogeneous chiral symmetry breaking phases,''
Phys. Rev. D \textbf{82}, 054009 (2010)
doi:10.1103/PhysRevD.82.054009
[arXiv:1007.1397 [hep-ph]].
%139 citations counted in INSPIRE as of 13 May 2020


%\cite{Bratovic:2012qs}
\bibitem{Bratovic:2012qs}
N.~M.~Bratovic, T.~Hatsuda and W.~Weise,
%``Role of Vector Interaction and Axial Anomaly in the PNJL Modeling of the QCD Phase Diagram,''
Phys. Lett. B \textbf{719}, 131-135 (2013)
doi:10.1016/j.physletb.2013.01.003
[arXiv:1204.3788 [hep-ph]].
%106 citations counted in INSPIRE as of 13 May 2020

%\cite{Fukushima:2013rx}
\bibitem{Fukushima:2013rx}
K.~Fukushima and C.~Sasaki,
%``The phase diagram of nuclear and quark matter at high baryon density,''
Prog. Part. Nucl. Phys. \textbf{72}, 99-154 (2013)
doi:10.1016/j.ppnp.2013.05.003
[arXiv:1301.6377 [hep-ph]].
%156 citations counted in INSPIRE as of 13 May 2020

%\cite{Hatta:2002sj}
\bibitem{Hatta:2002sj}
Y.~Hatta and T.~Ikeda,
%``Universality, the QCD critical / tricritical point and the quark number susceptibility,''
Phys. Rev. D \textbf{67}, 014028 (2003)
doi:10.1103/PhysRevD.67.014028
[arXiv:hep-ph/0210284 [hep-ph]].
%322 citations counted in INSPIRE as of 14 May 2020

%\cite{Stephanov:2008qz}
\bibitem{Stephanov:2008qz}
M.~Stephanov,
%``Non-Gaussian fluctuations near the QCD critical point,''
Phys. Rev. Lett. \textbf{102}, 032301 (2009)
doi:10.1103/PhysRevLett.102.032301
[arXiv:0809.3450 [hep-ph]].
%479 citations counted in INSPIRE as of 14 May 2020

%\cite{Berdnikov:1999ph}
\bibitem{Berdnikov:1999ph}
B.~Berdnikov and K.~Rajagopal,
%``Slowing out-of-equilibrium near the QCD critical point,''
Phys. Rev. D \textbf{61}, 105017 (2000)
doi:10.1103/PhysRevD.61.105017
[arXiv:hep-ph/9912274 [hep-ph]].
%258 citations counted in INSPIRE as of 15 May 2020

%\cite{Asakawa:2016dpt}
\bibitem{Asakawa:2016dpt}
M.~Asakawa, M.~Kitazawa, Y.~Onishi and M.~Sakaida,
%``Thermal blurring effects on fluctuations of conserved charges in rapidity space,''
Nucl. Phys. A \textbf{956}, 332-335 (2016)
doi:10.1016/j.nuclphysa.2016.02.042
%0 citations counted in INSPIRE as of 16 May 2020

%\cite{Mukherjee:2015swa}
\bibitem{Mukherjee:2015swa}
S.~Mukherjee, R.~Venugopalan and Y.~Yin,
%``Real time evolution of non-Gaussian cumulants in the QCD critical regime,''
Phys. Rev. C \textbf{92}, no.3, 034912 (2015)
doi:10.1103/PhysRevC.92.034912
[arXiv:1506.00645 [hep-ph]].
%88 citations counted in INSPIRE as of 16 May 2020


%\cite{Shovkovy:2012zn}
\bibitem{Shovkovy:2012zn}
I.~A.~Shovkovy,
%``Magnetic Catalysis: A Review,''
Lect. Notes Phys. \textbf{871}, 13-49 (2013)
doi:10.1007/978-3-642-37305-3-2
[arXiv:1207.5081 [hep-ph]].
%156 citations counted in INSPIRE as of 20 May 2020

%\cite{Fukushima:2012vr}
\bibitem{Fukushima:2012vr}
K.~Fukushima,
%``Views of the Chiral Magnetic Effect,''
Lect. Notes Phys. \textbf{871}, 241-259 (2013)
doi:10.1007/978-3-642-37305-3-9
[arXiv:1209.5064 [hep-ph]].
%55 citations counted in INSPIRE as of 20 May 2020

%\cite{Mueller:2014tea}
\bibitem{Mueller:2014tea}
N.~Mueller, J.~A.~Bonnet and C.~S.~Fischer,
%``Dynamical quark mass generation in a strong external magnetic field,''
Phys. Rev. D \textbf{89}, no.9, 094023 (2014)
doi:10.1103/PhysRevD.89.094023
[arXiv:1401.1647 [hep-ph]].
%54 citations counted in INSPIRE as of 20 May 2020

%\cite{Gursoy:2014aka}
\bibitem{Gursoy:2014aka}
U.~Gursoy, D.~Kharzeev and K.~Rajagopal,
%``Magnetohydrodynamics, charged currents and directed flow in heavy ion collisions,''
Phys. Rev. C \textbf{89}, no.5, 054905 (2014)
doi:10.1103/PhysRevC.89.054905
[arXiv:1401.3805 [hep-ph]].
%150 citations counted in INSPIRE as of 25 May 2020

%\cite{Kharzeev:2007jp}
\bibitem{Kharzeev:2007jp}
D.~E.~Kharzeev, L.~D.~McLerran and H.~J.~Warringa,
%``The Effects of topological charge change in heavy ion collisions: 'Event by event P and CP violation',''
Nucl. Phys. A \textbf{803}, 227-253 (2008)
doi:10.1016/j.nuclphysa.2008.02.298
[arXiv:0711.0950 [hep-ph]].
%1323 citations counted in INSPIRE as of 20 May 2020


%\cite{Skokov:2009qp}
\bibitem{Skokov:2009qp}
V.~Skokov, A.~Illarionov and V.~Toneev,
%``Estimate of the magnetic field strength in heavy-ion collisions,''
Int. J. Mod. Phys. A \textbf{24}, 5925-5932 (2009)
doi:10.1142/S0217751X09047570
[arXiv:0907.1396 [nucl-th]].
%722 citations counted in INSPIRE as of 20 May 2020

%\cite{Vachaspati:1991nm}
\bibitem{Vachaspati:1991nm}
T.~Vachaspati,
%``Magnetic fields from cosmological phase transitions,''
Phys. Lett. B \textbf{265}, 258-261 (1991)
doi:10.1016/0370-2693(91)90051-Q
%520 citations counted in INSPIRE as of 21 May 2020

%\cite{Grasso:2000wj}
\bibitem{Grasso:2000wj}
D.~Grasso and H.~R.~Rubinstein,
%``Magnetic fields in the early universe,''
Phys. Rept. \textbf{348}, 163-266 (2001)
doi:10.1016/S0370-1573(00)00110-1
[arXiv:astro-ph/0009061 [astro-ph]].
%757 citations counted in INSPIRE as of 21 May 2020

%\cite{Nahrgang:2011mg}
\bibitem{Nahrgang:2011mg}
  M.~Nahrgang, S.k~Leupold, C.~Herold and M.~Bleicher,
%  %``Nonequilibrium chiral fluid dynamics including dissipation and noise,''
  Phys.\ Rev.\ C {\bf 84}, 024912 (2011)
  doi:10.1103/PhysRevC.84.024912
  [arXiv:1105.0622 [nucl-th]].
  %%CITATION = doi:10.1103/PhysRevC.84.024912;%%
  %77 citations counted in INSPIRE as of 17 Jan 2019

%\cite{Fraga:2008qn}
\bibitem{Fraga:2008qn}
  E.~S.~Fraga and A.~J.~Mizher,
  %``Chiral transition in a strong magnetic background,''
  Phys.\ Rev.\ D {\bf 78}, 025016 (2008)
  doi:10.1103/PhysRevD.78.025016
  [arXiv:0804.1452 [hep-ph]].
  %%CITATION = doi:10.1103/PhysRevD.78.025016;%%
  %167 citations counted in INSPIRE as of 19 Mar 2020

%\cite{Menezes:2008qt}
\bibitem{Menezes:2008qt}
  D.~P.~Menezes, M.~Benghi Pinto, S.~S.~Avancini, A.~Perez Martinez and C.~Providencia,
  %``Quark matter under strong magnetic fields in the Nambu-Jona-Lasinio Model,''
  Phys.\ Rev.\ C {\bf 79}, 035807 (2009)
  doi:10.1103/PhysRevC.79.035807
  [arXiv:0811.3361 [nucl-th]].
  %%CITATION = doi:10.1103/PhysRevC.79.035807;%%
  %191 citations counted in INSPIRE as of 27 Jan 2020


%\cite{Berges:2001fi}
\bibitem{Berges:2001fi}
  J.~Berges,
  %``Controlled nonperturbative dynamics of quantum fields out-of-equilibrium,''
  Nucl.\ Phys.\ A {\bf 699}, 847 (2002)
  doi:10.1016/S0375-9474(01)01295-7
  [hep-ph/0105311].
  %%CITATION = doi:10.1016/S0375-9474(01)01295-7;%%
  %227 citations counted in INSPIRE as of 04 Feb 2019

%\cite{Fu:2010ej}
\bibitem{Fu:2010ej}
  W.~j.~Fu, D.~Huang and F.~b.~Wang,
  %``Time-dependent Ginzburg-Landau equation in the Nambu-Jona-Lasinio model,''
  Nucl.\ Phys.\ A {\bf 849}, 203 (2011)
  doi:10.1016/j.nuclphysa.2010.11.004
  [arXiv:1005.3636 [hep-ph]].
  %%CITATION = doi:10.1016/j.nuclphysa.2010.11.004;%%
  %4 citations counted in INSPIRE as of 11 Feb 2020

%\cite{Schwinger:1951nm}
\bibitem{Schwinger:1951nm}
J.~S.~Schwinger,
%``On gauge invariance and vacuum polarization,''
Phys.\ Rev.\  \textbf{82}, 664-679 (1951)
doi:10.1103/PhysRev.82.664
%4968 citations counted in INSPIRE as of 13 Apr 2020

%\cite{Chyi:1999fc}
\bibitem{Chyi:1999fc}
T.~Chyi, C.~Hwang, W.~Kao, G.~Lin, K.~Ng and J.~Tseng,
%``The weak field expansion for processes in a homogeneous background magnetic field,''
Phys.\ Rev.\ D \textbf{62}, 105014 (2000)
doi:10.1103/PhysRevD.62.105014
[arXiv:hep-th/9912134 [hep-th]].
%70 citations counted in INSPIRE as of 13 Apr 2020

%\cite{Hasan:2017fmf}
\bibitem{Hasan:2017fmf}
  M.~Hasan, B.~Chatterjee and B.~K.~Patra,
  %``Heavy Quark Potential in a static and strong homogeneous magnetic field,''
  Eur.\ Phys.\ J.\ C {\bf 77}, no. 11, 767 (2017)
  doi:10.1140/epjc/s10052-017-5346-z
  [arXiv:1703.10508 [hep-ph]].
  %%CITATION = doi:10.1140/epjc/s10052-017-5346-z;%%
  %21 citations counted in INSPIRE as of 06 Jan 2020

%\cite{Mallik:2009pj}
\bibitem{Mallik:2009pj}
  S.~Mallik and S.~Sarkar,
  %``Real-time propagators at finite temperature and chemical potential,''
  Eur.\ Phys.\ J.\ C {\bf 61}, 489 (2009)
  doi:10.1140/epjc/s10052-009-0990-6
  [arXiv:0901.1045 [hep-ph]].
  %%CITATION = doi:10.1140/epjc/s10052-009-0990-6;%%
  %29 citations counted in INSPIRE as of 13 Apr 2020

%\cite{Kubo:1957}
\bibitem{Kubo:1957}
R. Kubo, Journal of the Physical Society of Japan
12, 570 (1957).

%\cite{Kubo:1966}
\bibitem{Kubo:1966}
R. Kubo, Reports on Progress in Physics 29, 255 (1966)


%\cite{Binder:1981}
\bibitem{Binder:1981}
  K.~Binder,
  Z.\ Phys.\ B {\bf 43}, 119 (1981).
\end{thebibliography}

\end{document}

